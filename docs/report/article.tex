\documentclass[]{article}
\usepackage[english]{babel}
\usepackage{url}
\usepackage{graphicx}
\usepackage{hyperref}
\usepackage{listings}
\bibliographystyle{IEEEtranS}

% Title Page
\title{MCML with Boundary Surfaces}
\author{Marius Kircher}


\begin{document}
\maketitle

\begin{abstract}
	This project report describes an extended implementation of multi-layered Monte Carlo photon transport\cite{wang1992monte} (MCML). It supports structured boundaries, whereas original MCML only supported planar layers. This allows to use the method for simulation of light in more realistic materials. The report presents and compares radiance profiles of different boundary shapes. The focus lies on the properties reflectance and transmittance, that are important for visual appearance.
\end{abstract}

\section{Introduction}

The purpose of the MCML implementation presented here, is to provide a ground truth of surface appearance for physically based rendering. Monte Carlo methods are accepted to be leading in accuracy, but low-noise results have to be bought with long computation times. The implementation produces radiance profiles, that reveal the effect of varying surface shapes, and therefore allow to model surface appearance with approximate analytic curves. The curve equations, when found, can be used in faster rendering algorithms.

The report will first explain the principles of Monte Carlo light transport, before moving on to related work covering MCML specifically. On this basis, the contributions made by this project are stated and explained in detail in the implementation section. The new implementation is verified against the existing work and finally results are presented.

\section{Background}

\begin{figure}[ht!]
	\includegraphics[width=\linewidth]{img/trajectory.png}
	\caption{A single photon walk. Taken from~\cite{wang1992monte}.}
	\label{trajectory}
\end{figure}

Monte Carlo simulation of light is very accurate, because it simulates single photon packets and records the distribution of a large number of photons. This implies, that it is a statistical method, using a pseudo random number generator to compute, what can be thought of random walks of photons, visualized in \autoref{trajectory}.

A photon walk is defined by scattering and absorption in a specific material. These are also called the interaction properties:

\begin{description}
	\item[$\mu_s$] Scattering coefficient $[cm^{-1}]$ is the amount of scattering events per unit length
	\item[$\mu_a$] Absorption coefficient $[cm^{-1}]$ is the absorbed fraction of energy per unit length
\end{description}

In reality, these properties are also dependent on wavelength, which, in the quantum model, is equivalent to the amount of energy carried by one photon. In any case it means, that scattering and absorption are dependent on the color of the light. Since there is an infinite number of colors and natural appearance is the accumulation of subsets, computer graphics has to deal with composition from single colors. With the model used here, one can only simulate a single color at a time.

While absorption can be easily understood as energy that is transformed from light to heat, hereby leaving the considered system, scattering is more interesting. Different particles scatter light in different directions, that are generally hard to define. Simplified analytical definitions, called phase functions, are often used. They describe the angular probability distribution for photon direction changes. The popular Henyey-Greenstein phase function, that is used here, can be configured by scattering anisotropy $g$ as another material interaction property. A value of $g=0$ denotes a uniform probability distribution, i.e. isotropic scattering. Values of $g<0$ or $g>0$ control the amount of backward or forward scattering, respectively.

Finally, refractive indices are needed to compute the amounts of reflection and refraction at material boundaries according to Fresnel's formula.

\begin{figure}[ht!]
	\includegraphics[width=\linewidth]{img/layer-model.png}
	\caption{MCML's layer model. Taken from~\cite{wang1992monte}.}
	\label{layer-model}
\end{figure}

The material geometry in MCML is a sequence of layers as shown in \autoref{layer-model}. Photons enter through the coordinate origin and move in 3 dimensions.

Radiance distributions are recorded only as profiles, because the material model is radially symmetric. The reflectance profile is the distribution of photons that escaped at the top layer and transmittance is the analog at the bottom layer, in other words:

\begin{description}
	\item[$R_d(r)$] Diffuse reflectance $[J/cm^2]$ as function of radius $[cm]$
	\item[$T_t(r)$] Total transmittance $[J/cm^2]$ as function of radius $[cm]$
\end{description}

There are also other recorded quantities, such as absorption, that are of less interest in this report, as they do not contribute to visual appearance.

\section{Related Work}

\subsection{Light propagation model}

\begin{figure}[ht!]
	\includegraphics[width=\linewidth]{img/flowchart.png}
	\caption{Single layer Monte Carlo flowchart. Taken from~\cite{prahl89}.}
	\label{flowchart}
\end{figure}

The work is based on the light propagation model by Prahl et al.~\cite{prahl89}. You can get an overview of the algorithm by looking at the flowchart in \autoref{flowchart}.

Photons are initialized with a weight of $1.0$. The presence of weight means, that not actually single photons, but photon packets are simulated. A photon packet does not have atomic behavior, in the sense that its energy content can be divided into fractions. However the packet moves on a single path as if it was a single photon. This is done to reduce variance and hereby the overall number of photons needed for low-noise results. In the following, photon and photon packet refer to the same thing.

The photon performs a series of repeated steps. Each step length is generated by drawing a random number from a probability distribution based on the interaction coefficient $\mu_a+\mu_s$. The photon is moved by this length in the current direction. The moved photon path has to be checked for collision with the layer boundaries. After each step the photon interacts with the material.

If there was no collision, the photon drops a fraction of its weight based on $\mu_a$, which is added to total absorption. To avoid spending too much time simulating photons with small weight, i.e. small contribution to recorded quantities, but to still not bias the simulation, a so-called roulette is performed. Photons with very small weight enter the roulette. Such a photon dies with a probability $p \in (0,1)$ and if it survives, its weight is multiplied by $1/p$ as a countermeasure to the introduced bias. In the implementation described later, the weight threshold is $0.0001$ and $p=0.1$.

If the photon survives the roulette, its direction changes. A random number is drawn from the Henyey-Greenstein distribution, which yields the cosine of the deflection angle $\theta \in (0, \pi)$. This angle denotes how much the photon is diverted from its current path, i.e. is used for a rotation around an axis perpendicular to the current path. Relative to the current movement, $\theta$ can describe any change from $0$, i.e. forward scattering, to $\pi$, i.e. backward scattering. A second rotation around the axis of the current path uses a uniformly distributed angle $\psi \in (0, 2\pi)$.

If there was a collision, the algorithm decides if the photon is reflected, i.e. stays in the layer, or transmitted, i.e. leaves the medium. For this, the refraction angle is calculated using Snell's law and the refractive indices inside and outside the layer. Then Fresnel's formula is used to calculate the reflected fraction $R$. If for a uniform random number $\eta \in (0,1)$, $\eta > R$ is true, then the photon is transmitted, otherwise reflected. The algorithm (and also the implementation described later) does not perform partial reflection. If the photon leaves the layer, the remaining weight is added to the total reflection or transmission.

The simulation ends, when the target number of photons equals the number of photons, that left the layer or did not survive the roulette.

\subsection{Previous implementations}

Wang and Jacques~\cite{wang1992monte} published C source code implementing MCML in 1992 along with extensive documentation. The program is purely sequential and because of the inherently large number of photons required, it was worth working on another implementation.

Alerstam et al.~\cite{alerstam2008parallel} published a GPU implementation in 2008, which uses the Nvidia CUDA interface. The source code, user manual and implementation notes are available as download from the linked website, as of now. They achieved much faster simulations, but their program runs only on single Nvidia GPUs.

\section{Contribution}

The implementation of MCML presented here, that I want to call clustermcml, can use multiple PCs and GPUs by any vendor for parallelization and is based on OpenCL and MPI. Details are described in the implementation section.

The other contribution is support for non-planar layers. This is done by representing boundaries as radially symmetric heightfields. The symmetry allows to store only a 1-dimensional array of height samples. Non-symmetric heightfields would only make sense, once photon distributions are also recorded in two dimensions. Additionally, each height value is associated with a spacing, to enable irregular grids. The surface between height samples is defined by linear interpolation. In 3-dimensional space, a surface is therefore a sequence of connected concentric capped cones of finite height.

\section{Implementation}

\subsection{Parallelization}

Simulating individual photon packets is well suited for parallelization. The implementation does this on two levels.

On the first level, work is distributed onto multiple PCs that are connected by a local network. For this use case exists the Message Passing Interface (MPI), which provides a standardized C API and is based on inter-process communication. Before running MPI programs, one has to make sure, that if authentication is in place, all PCs can be accessed with the same username and password. Then the MPI listener service can be started on each node. The own program executable is passed to the MPI execution program, which connects to the listeners and starts a process on each node. MPI allows to query the number of nodes and the current node identifier, also called rank, from within the program. In general, all nodes use the same code for processing, but for I/O one node executes a special branch. It reads the input file and synchronizes data with an MPI Broadcast. All nodes work on the simulation and call the MPI Reduce function to compute the total sum of the resulting radiance arrays. Note that Reduce has to wait for all nodes to finish simulation. Finally, the I/O node writes the radiance data to the output file.

On the second level, work is further split on GPU threads. We chose OpenCL for this task, since it is available for all relevant GPU architectures. An OpenCL kernel implements the actual MCML photon transport for one photon packet. The maximum number of threads can be queried through OpenCL and depends on used registers and other GPU hardware resources. The kernel computes a fixed number of bounces to limit the time spent on the GPU, because the operating system would kill a process that blocks the GPU for too long. The program maintains an array by which it assigns one photon packet to one thread and repeatedly starts GPU runs, followed by checking the array for photon packets to be restarted until the target amount has been reached.

Work distribution is possible using different strategies. One implemented approach assumes equal processing power among MPI nodes and divides the total target number of photons up-front by the number of nodes, before each node starts its part of the simulation (note that we use the term photon, when we actually refer to photon packets, which is explained in \autoref{impl:photon-transport} about photon transport). In practice, this approach did not work well. Specifically, it sometimes caused MPI to abort because of a timeout during Reduce. Therefore a more fine-grained prallelization approach was implemented as follows.

First, we divide MPI nodes in several workers and one master. We can run a batch of $N$ photons in parallel, where $N$ equals thread count. In order to limit time spent on GPU, we always run for a fixed number of bounces. We cannot know how many runs it will take until the batch is finished (all photons terminated). Instead we count finished photons after each run. If after run $i$, a number $K > Z$ photons have finished, the worker sends $K$ to master. One thread of the master must be in receiving state. If the master finds that the number of photons that is not being worked on is $L >= K$, the response is CONTINUE, otherwise FINISH. When receiving CONTINUE, the worker spawns $K$ new photons, otherwise it finishes the remaining photons (with thread utilitization $< N$) and calls Reduce on its output arrays. When $L$ drops to zero, the master also calls Reduce, collects the total result and writes it to file.

After the master responds with FINISH, he finishes the remaining $L$ photons on his own, because there is no guarantee that a worker has space for exactly $L$ photons at some point, meaning (in the negative case) they could either run too many or $L$ never drops to zero.

With this approach, Reduce did not cause timeouts, since all workers receive a FINISH response within a short amount of time, because of the high communication frequency.

The high communication frequency, on the other hand, leads to a trade-off between thread utilization and communication efficiency, both impacting performance to yet unknown extent. If we set $Z$ to zero, we get maximum thread utilization, but high communication overhead. The higher $Z$ becomes, the worse thread utilization and the better communication efficiency becomes.

\subsection{MCML Kernel}
\label{impl:photon-transport}

The MCML kernel performs the actual photon transport. As input it receives an array of layers, where each layer contains interaction properties of its material, and an array of boundaries containing the geometry information. In contrast to existing MCML implementations, it seemed reasonable to split these two classes of information, because supporting different boundary surfaces requires more involved geometric data and operations.

The output consists of three radiance arrays for diffuse reflectance, absorbance and transmittance. The arrays can be thought of histograms over radius with $n_r$ bins of size $d_r$ centimeters, respectively. A second dimension can be added by specifying a number of depth bins $n_z > 1$ of size $d_z$ for the absorption array, or a number of angular bins $n_a > 1$ for the reflectance and transmittance arrays. See the MCI file format in~\cite{wang1992monte}.

A photon packet is defined as having an energy weight of $1.0$ when it is launched. The weight may be reduced multiple times by absorption during the walk, before the remaining fraction is added to a total radiance, which is finally normalized by the number of photon packets simulated. This reduces variance, compared to a simulation of photons, that really behave atomically. The packet, however, behaves like a single photon in the sense that it has a single path.

When a boundary collision is detected, the photon moves a shorter path than the previously calculated step length. The original MCML implementation saves the remaining step length for the next step, but CUDAMCML authors argue, that the randomized next step length is independent of its history. The implementation presented here therefore also does not save the remaining step length, but ignores it.

An additional data structure is the array of photon states, which is needed to preserve state across GPU runs. A photon state contains current position, direction, layer index, weight and state of the random number generator.

A xorshift~\cite{marsaglia2003xorshift} random number generator is used. It takes more processor instructions, but produces higher quality pseudo random numbers than the linear congruential generator, which was used in the original MCML.

When a photon state is initialized, a hash function is used on the state array index to seed the xorshift generator. This ensures well distributed seeds and hereby largely independent random sequences. Care has to be taken, to not seed xorshift with zero, since the sequence would stay zero forever.

\subsection{Boundary Specification}

Previous MCML implementations used flat boundaries, that were implicitly defined by the layers. The implementation presented here supports both implicit boundaries and explicitly defined heightfields.

Heights are interpreted from heightfield origin, which is defined by layer thicknesses, into negative z-direction, because in MCML the z-axis points \emph{down} into the material. The last height sample is valid for all radii that are out of bounds.

To enable mixing implicit and explicit boundaries in the same simulation, the MCI file format was extended. Lines containing explicit boundary data start with 'b', then the number of samples and then the list of sample values, which consists of alternating heights and spacings. This is an excerpt from an example MCI file:

\begin{lstlisting}
[...]

2 # Number of layers
1 # n for medium above

# explicit heightfield boundary with 2 samples
b 2 0.2 0.1 0.0 0.1

# n mua mus g d
1.4 0.1 90  0 0.5 # layer 1

# implicit flat boundary

1.4 0.1 90  0 0.5 # layer 2

# explicit heightfield boundary with 3 samples
b 3 0.2 0.1 0.1 0.1 0.0 0.1

1 # n for medium below
\end{lstlisting}

Old formats are supported by the new implementation.

Explicit heightfield boundaries can also be turned on or off globally for performance reasons. When the kernel define IGNORE\_HEIGHTFIELDS is set, any explicit boundary data from the file is ignored.

The recorded radiance is clamped, so that all photon packets that fall outside the array bounds are accumulated at the edges. In original MCML this could only happen in xy directions. Now also the depth index of the absorption detection array must be clamped, as we can have for example weight drops at a depth less than zero.

The implementation checks if heightfields overlap each other before starting the simulation. Overlapping heightfields constitute an invalid layer model.

\subsection{Boundary Intersection Test}

After each step of a photon packet, the moved line has to be tested for an intersection with two radial heightfields, the one at the top of the current layer and the one at the bottom.

The base of the heightfield intersection algorithm is an analytical method for intersecting rays with cones by David Eberly\cite{schneider2002geometric}. The cone shapes arise when the radial 1D heightfield is rotated in the xy-plane around the origin. The algorithm is outlined in the following.

First we must find all potential cones in the path of the line. We assign an ascending index from inner to outer cones. To find the highest candidate index, we count the spacings, that fit into the length of the xy-vectors of the start and end of the line, and take the maximum. The lowest candidate index is obtained by projecting the heightfield origin onto the line and count the spacings that fit into the length of the xy-vector of the resulting point. All cones between these two indices, inclusively, are tested for intersection with the line and the closest hit is returned.

To calculate reflections and transmissions at heightfield boundaries, a normal is needed. The line-cone intersection method was extended to additionally return the normal at the point of intersection. There are sharp local minima and maxima where the gradient cannot be defined, since the heightfield is interpolated linearly. For these cases the boundary is treated as a xy-plane with normal along z-axis. The direction of the normal is always chosen to point into the half space from where an intersecting ray comes in.

If there was an intersection in the last photon step and the photon is still inside the simulation domain, the corresponding boundary has to be temporarily disabled for intersections during the next step. The reason for this is that the photon might not be moved exactly onto the boundary surface, because of limited floating point accuracy, and it would therefore be possible to detect another erroneous intersection with the same boundary.

There is a fallback to a faster intersection test with a plane, if a boundary is not a heightfield.

\subsection{Building and Running}

When building the program for the first time on a new computer, you have to set the paths to your compiler, OpenCL and MPI libraries in the Makefile. Then use nmake to build. For execution, use the following command:

\begin{lstlisting}
mpiexec /machinefile mpi.txt /pwd "***"
clustermcml-windows.exe mcmlKernel.c "-Werror" sample.mci
\end{lstlisting}

where the machinefile mpi.txt specifies the IP address and number of processes per MPI node. The executable is called clustermcml-windows and receives as first argument the MCML kernel file, which will be compiled by the OpenCL implementation. After the kernel file follow the OpenCL compiler arguments, where this example wants to print all warnings as errors. On the tested machines, the current kernel code did not generate warnings. You can add other compiler flags, such as "-D IGNORE\_A" to skip writing to absorption array or "-D IGNORE\_HEIGHTFIELDS" to use implicit flat boundaries, where both options exist to speed up simulation times.

There is another make target for an executable called singlemcml-windows, which is a non-parallel and CPU-only version of the program for the purpose of easier debugging.

The implementation was currently only built and run on Microsoft Windows, but all used libraries are also available on other platforms. There is no build system other than make (on Windows: nmake) in place, because the project is relatively small and simple. The Makefile can be altered with analog compiler commands and flags, as well as runtime and library paths.

\subsection{Limitations}

There are minor deviations from original MCML. The following points are very similar to the limitations of CUDAMCML, as can be seen in their manual in section 2.4.

\begin{itemize}
\item The binary output is currently not supported. MCO files are always written as ASCII.
\item Glass layers are not supported. Reflectance will always be reported as diffuse.
\item The maximum number of photons is limited to $2^{32}-1$. Also the number of photons must be higher than the number of GPU threads, which will be printed on program startup.
\item Input data is not checked for sanity.
\item All calculations are preformed with 32-bit floating point precision. This is also the case for CUDAMCML and was shown to have only slight effect.
\end{itemize}

The implementation currently leaves much room for performance optimization. Important aspects are listed here.

\begin{itemize}
\item The array of photon states is iterated sequentially after each GPU run, to check for photons to restart. This operation can be parallelized or integrated into the already parallel kernel. CUDAMCML restarts finished photons immediately from the kernel. This also avoids inactive threads, but introduces the cost of another atomic counter to keep track of total finished photons.
	
\item GPUs have global and local memory spaces, where access to local memory is usually faster. Therefore, CUDAMCML accumulates weights in local variable before performing an atomic add on a global output array. The implementation presented here performs the atomic add every time, when a weight is stored.
	
\item To read a height value from a boundary at arbitrary floating point coordinates, the heightfield must be interpolated. Therefore it would be efficient to store the heightfields in textures and use built-in access operators on the GPU. Currently, the program uses raw buffers and does interpolation in code.
\end{itemize}

\section{Verification}

\autoref{verification1} compares the diffuse reflectance profiles of clustermcml and CUDAMCML using 100 million photons and one layer of thickness $dl=0.1cm$. The absorption coefficient is $\mu_a=0.1$, the scattering coefficient $\mu_s=1$ and scattering anisotropy $g=0.5$. The refractive indices are $n=1.34$ inside the layer and $1$ outside. The detection window has a size of $1cm$ and a resolution of $0.005cm$. Note that a logarithmic scale is used for a quantity, that is reflectance multiplied by radius. The plots indicate correct results despite different amounts of noise.

\begin{figure}[ht!]
	\includegraphics[width=\linewidth]{img/verification1.pdf}
	\caption{Comparison of the diffuse reflectance profiles obtained with the implementation presented in this report, clustermcml, and the previous implementation CUDAMCML as reference.}
	\label{verification1}
\end{figure}

\section{Results}

This section shows novel radiance profiles that arise from simulations with different boundary surface shapes.

\begin{figure}[ht!]
	\includegraphics[width=\linewidth]{img/result1.pdf}
	\caption{Comparison of diffuse reflectance profiles of 7 different surfaces.}
	\label{result1}
\end{figure}

\begin{figure}[ht!]
	\includegraphics[width=\linewidth]{img/result2.pdf}
	\caption{Comparison of diffuse reflectance profiles of flat surface with 4 variations of curved surfaces.}
	\label{result2}
\end{figure}

\begin{figure}[ht!]
\includegraphics[width=\linewidth]{img/result3.pdf}
\caption{Comparison of diffuse reflectance profiles of flat surface with 2 variation of noise surfaces.}
\label{result3}
\end{figure}

\section{Conclusion and Future Work}

\bibliography{bibliography}
\end{document}          
